\documentclass[a4paper,oneside,12pt]{book}
%% === nezbytné balíčky:
\usepackage[IL2]{fontenc}   % font for UTF-8 || other: [T1] (works for UTF-8?)
\usepackage[utf8]{inputenc} % input archive UTF-8 || [cp1250] -> Windows 1250 || [latin2] -> ISO Latin 2

\usepackage[english]{babel} % english wrote work, typographical rules     

\usepackage{pdfpages} % package allows to use pdf file insertion. Used to insert zadani_cele.pdf

\usepackage[a4paper, hmarginratio=1:1]{geometry} % usage of full A4 document page and frame limit - for two sided printing

%% === suitable packages
\usepackage[hidelinks]{hyperref} % a package - there will be clickable links in the PDF
\usepackage{graphicx} % a package for the insertion of RASTER graphics (PNG, etc.)
\usepackage{epsfig} % a package for the insertion of EPS-type VECTOR graphics files

\usepackage{float} % extended image location options
\usepackage{caption} % for the description of pictures, tables, etc.

\usepackage{tabularx} % extended table options
%\usepackage{tabu} % other package for extended spreadsheet options


\usepackage{listings}  % Package suitable for code samples 
\usepackage{amsmath} % the advanced mathematical rate package
\usepackage{color} % for the possibility of coloured text
\usepackage{fancybox} % Allows for advanced framing
%\usepackage{index} % to be used in case of makeindex registry creation. !!READ NEXT LINE
%\newindex{default}{idx}{ind}{Rejstřík} % establishes a register in the case of the use of an index package

%By me
\usepackage{fixltx2e} %Allows to use subscript for text

\topmargin=-15mm      % the upper edge slightly smaller
\textwidth=150mm      % width of the text on the page
\textheight=240mm     % 'height' of the text on the page

\nonfrenchspacing % In english texts space after sentence is bigger
\widowpenalty=1000 % "power" of the widow ban (= one line of paragraph at the bottom of the page)
\clubpenalty=1000 % "power" orphan ban (= one line/word paragraph separately at the top of the page)
\brokenpenalty=1000 % "power" ban of breaking the page after a line which has a split word at the end

\pagenumbering{arabic} % page numbering by Arabic numerals
\pagestyle{plain}      % pages numbered from the bottom to the middle

\parindent=0pt % deleting the first line of paragraph 1
\parskip=7pt   % gap between paragraphs

\newcommand{\ti}{\textit} % abbreviated command for italics
\newcommand{\tb}{\textbf} % abbreviated command for bold


%%-- there are macros, or "constants"-- some of them are necessary to be changed! --- %%
\newcommand{\cvut}{Czech Technical University in Prague}
\newcommand{\fjfi}{Faculty of Nuclear Sciences and Physical Engineering}
\newcommand{\katedra}{ Department of Physics}
\newcommand{\program}{Nuclear and Particle Physics} % stud. program
\newcommand{\spec}{--} % if a program has specialisation

\newcommand{\druh}{Bachelor's thesis}
\newcommand{\woman}{} % if a woman change to \newcommand{\woman}{a}

\newcommand{\logoCVUT}{\includegraphics{symbol_cvut_konturova_verze_cb.pdf}} % the logo of the CTU -- according to the CTu graphic manual in force since December 2016. If the PDF-version does not fit, use another version of the logo: https://www.cvut.cz/logo-a-graficky-manual -> "Symbol and logo of the CTU in Prague"). If you want to omit the logo altogether, instead of "\includegraphics{...}" type the text "\vspace{35mm}"

% exactly according to the form 'Application for a Bachelor's degree/degree':
\newcommand{\nazevcz}{Tvrdé sondy ve vysokoenergetických srážkách na RHIC}    % Czech job title (according to the assignment!)
\newcommand{\nazeven}{Hard probes in high energy collisions at RHIC}          % English job title (according to the assignment!)
\newcommand{\autor}{Nikolai Denisov}
\newcommand{\vedouci}{Dr. Barbara Antonina Trzeciak, Ph.D.} % the name and surname of the supervisor of the work, including titles, e.g.: Doc. Ing. Ivo Malý, Ph.D.
\newcommand{\pracovisteVed}{\cathedra, \FNSPE, \CTU}
\newcommand{\konzultant}{--} % If there is a designated consultant, enter his name + titles
\newcommand{\pracovisteKonz}{--} % If there is a designated consultant, enter his work place

% Fill accordingly:
\newcommand{\rok}{2024}  % year of submission (only the year of submission, not the entire academic year!)
\newcommand{\kde}{Prague}

\newcommand{\klicova}{Jets, Quark guon plasm, jet algorithms, probing, D$_0$}   % Here type in czech about 3-5 keywords !ASK
\newcommand{\keyword}{Key words}       % here type in English about 3-5 key words (translate from Czech, but professionally)

\newcommand{\abstrCZ}{Thesis description in Czech}    % Write abstract in Czech (at least 7 sentences, min. 80 words) here. Ensure that both the CZ and EN abstracts do not cause page 6 to overflow to page 7, i.e. that both with keywords fit on ONE page) !ASK
\newcommand{\abstrEN}{Probing quark guon plasm is not trivial task due to its period of existance. There are different methods to do that. One of the most common methods to probe and study its hard properrties is through hard collisions - high momentum trasver. Such transfer can occure in jet formation. In that work simulated data with Pythia8 were compared to data produced on RHIC and LHC to evaluate if it reasanoble for hard probes.} % Write abstract in English !ASK

\newcommand{\prohlaseni}{I declare that I have prepared my thesis on my own and have used only the materials (literature, projects, etc.) listed in the attached list.}

\newcommand{\podekovani}{Thanks you... for...} % WRITE a thank-you note, for example to your supervisor !ASK


\begin{document}
%%%%%%%%%%% TITLE -- the next 30 or so lines are generated AUTOMATICALLY. Do not change!!! Exception: papers written in English! %%%%%%%%%%%%
    \thispagestyle{empty}

    \begin{center}
    {\Large \textsc{\cvut}\\[4mm] \textsc{\fjfi}}\par
    \vspace{4mm}
    \tb{\katedra} \par\vspace{3mm}

    \begin{tabular}{l}
        \tb{Study program: \program}\\
        \tb{Specialization: \spec}\\
    \end{tabular}

    \vspace{10mm} \logoCVUT \vspace{15mm}

    {\huge \tb{\nazevcz}\par}
    \vspace{5mm}
    {\huge \tb{\nazeven}\par}

    \vspace{15mm}
    {\Large \MakeUppercase{\druh}}

    \vfill
    {\large
        \begin{tabular}{ll}
            Produced by: & \autor\\
            Supervisor: & \vedouci\\
            Year: & \rok
        \end{tabular}
    }
    \end{center}

%%%%%%%%%%%% ZADÁNÍ PRÁCE %%%%%%%%%%%%
% Zadání (podepsané děkanem atd.) dostanou studenti KSI od sekretářky (nebo ji požádají, např. e-mailem).
    \newpage  % SEM NESAHEJTE!
    \thispagestyle{empty} % SEM NESAHEJTE!

    \includepdf[pages={1,2}]{zadani_cele.pdf} % PDF má 2 stránky



%%%%%%%%%%%% Prohlášení -- SEM NESAHEJTE! Generuje se automaticky z výše nastavených maker \kde{} a \prohlaseni{}. %%%%%%%%%%%%
    \newpage % SEM NESAHEJTE!
    \thispagestyle{empty}  % SEM NESAHEJTE!

    ~ % SEM NESAHEJTE!
    \vfill % prázdné místo. SEM NESAHEJTE!

    \tb{Declaration} % SEM NESAHEJTE!

    \vspace{1em} % vertikální mezera. SEM NESAHEJTE!
    \prohlaseni

    \vspace{2em}  % SEM NESAHEJTE!
    \hspace{-0.5em}\begin{tabularx}{\textwidth}{X c}  % SEM NESAHEJTE!
                       In \kde\  .................... &........................................ \\	% SEM NESAHEJTE!
                       & \autor
    \end{tabularx}	% SEM NESAHEJTE!


%%%%%%%%%%%% Poděkování -- tuto stránku můžete celou odstranit %%%%%%%%%%%%
    \newpage
    \thispagestyle{empty}

    ~
    \vfill % prázdné místo

    \tb{Poděkování}

    \vspace{1em} % vertikální mezera
    \podekovani
    \begin{flushright}
        \autor
    \end{flushright}  % <------- tady končí stránka s poděkováním


%%%%%%%%%%%% ABSTRAKT atp. Je generován AUTOMATICKY podle maker nastavených na začátku souboru) %%%%%%%%%%%% 
    \newpage   % SEM NESAHEJTE!
    \thispagestyle{empty}   % SEM NESAHEJTE!

% příprava:    (na následujících 8 řádků NESAHEJTE!)
    \newbox\odstavecbox
    \newlength\vyskaodstavce
    \newcommand\odstavec[2]{%
        \setbox\odstavecbox=\hbox{%
            \parbox[t]{#1}{#2\vrule width 0pt depth 4pt}}%
        \global\vyskaodstavce=\dp\odstavecbox
        \box\odstavecbox}
    \newcommand{\delka}{120mm} % šířka textů ve 2. sloupci tabulky

% použití přípravy:    % dovnitř "tabular" vůbec NESAHEJTE!
    \begin{tabular}{ll}
    {\em Název práce:} & ~ \\
    \multicolumn{2}{l}{\odstavec{\textwidth}{\bf \nazevcz}} \\[1em]
    {\em Autor:} & \autor \\[1em]
    {\em Studijní program:} & \program \\
    {\em Specializace:} & \spec \\
    {\em Druh práce:} & \druh \\[1em]
    {\em Vedoucí práce:} & \odstavec{\delka}{\vedouci} \\
    {\em Konzultant:} & -- %\odstavec{\delka}{\konzultant \\ \pracovisteKonz}  % VYMAŽTE text "-- %" v případě, že jste neměli konzultanta
    \\[1em]
    \multicolumn{2}{l}{\odstavec{\textwidth}{{\em Abstrakt:} ~ \abstrCZ  }} \\[1em] %WRITE ABSTARCAT
    {\em Klíčová slova:} & \odstavec{\delka}{\klicova} \\[2em]
    %WRITE KEY WORDS
    {\em Title:} & ~\\
    \multicolumn{2}{l}{\odstavec{\textwidth}{\bf \nazeven}}\\[1em]
    {\em Author:} & \autor \\[1em]
    \multicolumn{2}{l}{\odstavec{\textwidth}{{\em Abstract:} ~ \abstrEN  }} \\[1em]
    {\em Key words:} & \odstavec{\delka}{\keyword}
    \end{tabular}



%%%%%%%%%%%% Obsah práce ... je generován AUTOMATICKY %%%%%%%%%%%%
    \newpage  % SEM NESAHEJTE!
    \parskip=0pt
    \tableofcontents % SEM NESAHEJTE!
    \parskip=7pt
    \newpage % SEM NESAHEJTE!


%--------------------------------------------------------
%|         Here starts the thesis (text)              |
%--------------------------------------------------------

    \chapter{Introduction} %add quark guon plasma, phase diagram, probbing with focus on jets and heavy flavour
    \section{General description} %, rapidity, time-space diagram, jet variables, CAN't BE LIKE THAT MUT BE MORE DESCIBETIVE
    \section{Physics of the particle collisions}
    \section{Current results on heavy flavour jet angularities} %describe results from ALICE
    \section{RHIC and STAR}



    \chapter{Jet algorithms} %brief description
    \section{k\textsubscript{t} algorithm}
    \section{anti-k\textsubscript{t} algorithm}
    \section{Cambridge-Aachen algorithm} %brief
    \section{SISCone Algorithm algorithm} %brief


    \chapter{Simulations of hard probes in high energy collisions}
    \section{Pythia}
    \section{STAR and RHIC settings}
    \section{Simulation results} %algorithms comparision, basic distributions with star requirments

    \chapter{Heavy flavour jets simulations for STAR}
    \section{D\textsubscript{0} jet tagging}
    \section{Variables}









    \chapter*{Conclusion} %What have been simulated, for what and why Pythia is good for that

    \addcontentsline{toc}{chapter}{Conclusion}

%%%%%%%%%%%% The list of used sources (LITERATURE) %%%%%%%%%%%% !ASK IF IT IS POSSIBLE TO ADD PRESENTAtIONs -> NO, but I can make reference to one of the books 
    \clearpage
    \addcontentsline{toc}{chapter}{The list of used sources}
    \begin{thebibliography}{99}
% replace the following text (2 lines) with citations of your sources
        \bibitem{odkaz} Autor. \ti{Název knihy}. Město. Nakladatelství. Rok.
    \end{thebibliography} %ADD GITHUB HERE
% the following format: CN ISO 690. You can generate it at http://www.citacepro.com (sign in via the "CITATES" link), it can generate TeX
% order: alphabetically by author (or by first name, if author is unknown)


    \newpage
    \addcontentsline{toc}{chapter}{Apendix} % Pythia settings for visualisation, for STAR, for D_0 jets, additional plots. NO CODE
    \appendix

    \chapter{Apendicies' name} % here change the name of the annex, or comment and insert the file where the name of the annex is + its text (delete/comment the text below + uncomment \input{}):

\end{document} % End.
