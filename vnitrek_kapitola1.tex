\small % zmenšení velikosti písma, aby pokyny byly kratší :-)
Tento text, vložený do \uv{šablony} příkazem \texttt{\textbackslash input\{vnitrek\_kapitola1.tex\}}, slouží jako stručné pokyny\footnote{Podrobnější informace o~psaní závěrečné práce získáte v~rámci předmětu \uv{Seminář k~bakalářské práci} (resp. \uv{Seminář k~diplomové práci}).} pro psaní závěrečné práce a~zároveň jej můžete využít jako ukázku práce v~\LaTeX{u}.
\par 
Ve skutečné závěrečné práci nahraďte soubor \texttt{vnitrek\_kapitola1.tex} za jiný (nebo kompletně přepište obsah souboru).

\newcommand{\prikaz}[1]{\texttt{\textbackslash #1\{\}}} % pomocné makro, definice by měla být spíše v preambuli, ale jelikož slouží pouze pro kapitolu~1 (a~ne pro všechny studenty), tak je umístěno zde

 
\section{Obsah textu}  
Obsah závěrečné práce musí zahrnovat všechny body z~osnovy (formulář \uv{Zadánín bak./dipl. práce}). Rozšíření zadání práce není zakázáno.
\par 
Textová část práce musí být {\em autorským dílem studenta} -- vyhýbejte se proto jakýmkoliv pokusům o~plagiátorství. Použité zdroje (knihy, webové stránky, software apod.) by měly být řádně citovány! Seznam použitých zdrojů se uvádí na konci práce, jednotlivé zdroje se citují příkazem \prikaz{cite}. Vyhněte se nejednoznačnostem a~nejasnostem v~textu.
\par
Teoretická část práce bývá v~rozsahu alespoň 10~stran. Z~textu musí být jasně patrné, co je převzato z~literatury a~co jsou původní myšlenky studenta. Teorie by měla souviset s~výsledkem závěrečné práce.
\par 
Praktická část práce by měla popsat/obsahovat dosažené výsledky (např. funkční program, číselný výsledek řešení, nové pravdivé řešení dokázané v~rámci teorie, objektivně zdůvodněné rozhodnutí). Konkrétní výsledek je zvýrazněn v~závěru práce a~bez teorie uvedené v~práci ho není možné dosáhnout. Výsledky práce by měly přinášet nějaké nové informace a~vlastní pohled studenta na zpracovávané téma.
\par
Samotná práce by měla být dobře strukturovaná a její jednotlivé části by na sebe měly {\em dobře logicky navazovat}. Důraz by měl být kladen na věcnost, odbornou úroveň, dobrou čtivost a~(je-li to možné) aktuálnost textu. Velmi důležitá je {\em faktická přesnost a~úplnost} (zejména při použití již neaktuálních zdrojů).
\par
{\bf Tip:} vyhněte se používání nevysvětlených zkratek. Nepoužívejte \uv{odbornou hantýrku}. Vyhněte se výrokům, které se dají různě interpretovat. Nepoužívejte příliš dlouhé věty. 



\section{Členění textu do kapitol a~sekcí}
Závěrečná práce obvykle obsahuje asi 5-6 kapitol (včetně úvodu a~závěru, tj. očíslované kapitoly jsou obvykle tři až čtyři). Kapitoly lze rozdělit na podkapitoly a~tyto dále na jednotlivé sekce:
\begin{itemize}
\item kapitola: {\texttt{\textbackslash chapter\{Název\}}},
\item podkapitola (v~kapitole): {\texttt{\textbackslash section\{Název\}}},
\item sekce (v~podkapitole):  {\texttt{\textbackslash subsection\{Název\}}}.
\end{itemize}
Dělení do nižších úrovní se nepoužívá (tak velký rozsah závěrečné práce nemají, aby byla potřeba další úroveň obsahu). 
\par
Názvy (pod)kapitol a~sekcí píšeme malými písmeny, kromě velkých začátečních písmen a~jmen (např. \uv{Trh v~České republice}). V~anglicky psané práci jsou pravidla jiná!
\par 
{\bf Tipy:} 
\begin{enumerate}
\item Je vhodné uložit si texty jednotlivých kapitol (nebo i~podkapitol) do samostatných souborů, neboť pak lépe/jednoduše přeskupíte vybrané části, když zjistíte, že původní obsah práce nebyl navržen vhodně. \par Typickým znakem nevhodně členěné práce jsou \uv{malé} (cca čtyřstránkové) kapitoly a~mezi nimi jedna \uv{velká} 20stránková kapitola.
\item Zdá se, že potřebujete čtvrtou úroveň členění? Zvažte, zda z~nějaké podkapitoly neudělat novou kapitolu závěrečné práce.
\item Pokud nevíte, jak vhodně rozčlenit svůj text, {\em požádejte o~radu vedoucího\/} své práce (při pravidelných konzultacích).
\end{enumerate}

\section{Obrázky a tabulky}\label{sekceObrTab}

Text závěrečné práce je vhodné doplnit tabulkami nebo obrázky (grafy, diagramy užití apod.), abyste lépe ilustrovali své myšlenky nebo dosažené výsledky. 
\par 
Obrázky a~tabulky se vkládají do plovoucího prostředí (\texttt{figure}, \texttt{table}). {\color{red} Plovoucí prostředí si LaTeX umístí sám} (na typograficky vhodné místo -- typicky nahoru či naopak dolů na stránku. Občas se obrázky dostanou až na konec kapitoly, ale nejedná se o~chybu).
\par 
Popis obrázku/tabulky se píše do příkazu \prikaz{caption} a~měl by být stručný a {\em výstižný}. Obrázky mají popis dole, kdežto tabulky mají popis nahoře. Pokud je obrázek (resp. tabulka) převzatý, nezapomeňte na konci popisku citovat zdroj!
\par 
Do textu vždy {\em napište důvod použití\/} obrázku/tabulky a~odkažte se na něj/ni:
\begin{itemize}
\item vytvoření odkazu: příkaz \prikaz{label} následovaný po příkazu \prikaz{caption},
\item sazba čísla odpovídajícího odkazu (tj. např. číslo obrázku): příkaz \prikaz{ref},
\item sazba čísla stránky, kde je odkaz: příkaz \prikaz{pageref}.
\end{itemize}   
Tabulky a~obrázky v~plovoucím prostředí používají své vlastní, {\em nezávislé číselné řady}. Z~toho vyplývá, že v~odkazech uvnitř textu musíme kromě čísla udat i~informaci o~tom, zda se jedná o~obrázek, anebo tabulku! Příklad: \texttt{vizte obrázek\textasciitilde\textbackslash ref\{odkazObrA\}}.
\par

\subsection{Obrázky}
Obrázky se vkládají do plovoucího prostředí \texttt{figure}. Nesnažte se přesvědčit \LaTeX, aby byl obrázek umístěný uprostřed stránky! Porušilo by to typografická pravidla sazby.\par 
Formát obrázku volte pokud možno vektorový (tj. EPS nebo PS), protože se dají zvětšovat i~zmenšovat bez újmy na čitelnosti. Pokud máte k~dispozici obrázek pouze v~rastrovém formátu (např. PNG nebo schéma v~JPG), nakreslete jej znovu v~některém vektorovém grafickém programu. Nemá smysl konvertovat bitmapové obrázky do formátu EPS.
\par 
{\bf Tip:} pomocí balíčku \href{https://texample.net/}{\underline{TikZ}} lze kreslit některé typy obrázků přímo v~textu práce. (Pro použití TikZ je potřeba přidat správný příkaz \prikaz{usepackage} v~preambuli \uv{šablony}.)

\subsubsection*{Ukázka -- kód pro vložení obrázku}
Příkazy pro vložení obrázku \texttt{graf.png}, na který se bude dát odkazovat pomocí \texttt{grafA}:
\begin{verbatim}
\begin{figure}
  \centering      % vycentrovat
  \includegraphics[scale=1]{graf.png} % soubor + měřítko (scale)
  \caption{Graf závislosti $y$ na $x$.} % popis obrázku
  \label{grafA} % definice odkazu na obrázek (pro \ref{})
\end{figure}
\end{verbatim}

\subsection{Tabulky} 
Tabulky se vkládají do plovoucího prostředí \texttt{table}. Součástí tohoto dokumentu je tabulka~\ref{tabMakra} (na straně~\pageref{tabMakra}), která obsahuje přehled maker ze \uv{šablony}.

\subsubsection*{Ukázka -- kód pro vložení tabulky}
Příkazy pro dvousloupcovou tabulku se záhlavím (tučně) a jedním řádkem, vše orámováno čarou (styl orámování tabulky si můžete zvolit, ale dodržujte stejný v~celé práci): 
\begin{verbatim}
\begin{table}
  \centering
  \caption{Popis tabulky.} 
  \label{odkazTabA} % odkaz na číslo tabulky, lze využít v \ref{odkaz}
  \begin{tabular}{|l|l|} % 2 sloupce, zarovnání obsahu doleva
    \hline \bfseries{název1} & \bfseries{název2}\\
    \hline data1 & data2 \\
    \hline % koncová čára
  \end{tabular}
\end{table}
\end{verbatim}


\section{Typografická pravidla}\label{podkapTypo}
Text práce musí vyhovovat typografickým pravidlům. Pokud netušíte, o~co jde, vyhledejte si např. heslo \uv{Základní pravidla hladké sazby}.
\par 
Především se zaměřte na:
\begin{itemize}\itemsep=2pt
\item {\bf členicí (interpunkční) znaménka} $\rightarrow$ podrobnosti pod heslem \uv{Pravopis -- interpunkce} na webu \url{https://prirucka.ujc.cas.cz/}.\par
Základní typografická pravidla pro znaménka:\\
Tečka, čárka, středník, dvojtečka, otazník a vykřičník se přimykají k~předcházejícímu slovu bez mezery (mezera se píše až za nimi). Výjimkou je desetinná čárka (v~anglicky psané práci desetinná tečka), okolo které se mezery nepíšou;
\item {\bf spojovník} (spojovací čárka) a~{\bf pomlčka}\footnote{Pomlčka v~anglicky psané práci se sází jako \texttt{-{}-{}-} a bez mezer okolo.} jsou různé znaky! Pro spojovací čárku píšeme znak \texttt{-}, kdežto pomlčku sázíme jako \texttt{-{}-}.\par 
Kdy (a~jak) se používá pomlčka: \url{https://prirucka.ujc.cas.cz/?id=165}.\par 
Kdy se používá spojovník: \url{https://prirucka.ujc.cas.cz/?id=164};
\item {\bf závorky} se přimykají k~vnitřnímu textu, tedy mezera se píše  před levou závorkou, a~pak za pravou závorkou;
\item \textbf{uvozovky} se přimykají k~vnitřnímu textu. V~česky psané práci můžete využít makro {\tt \textbackslash uv\{text\}}. 
\par
Pozor: pokud uvádíte ukázky zdrojového kódu programů, tak v~nich se používají uvozovky anglické (a také jiný typ písma).
\par Podrobnosti k~používání uvozovek: \url{https://prirucka.ujc.cas.cz/?id=162};
\item \textbf{apostrof} (odsuvník): \url{https://prirucka.ujc.cas.cz/?id=168};
\item \textbf{lomítko} jako oddělovač se píše bez mezer. Příklad: {\tt školní rok 2000/2001}; %
\item \textbf{procento} se sází se zpětným lomítkem. Pozor na rozdíl mezi 20\,\% (dvacet procent) a 20\% (dvacetiprocentní)! Podrobnosti: \url{https://prirucka.ujc.cas.cz/?id=790};
\item \textbf{pevná mezera} (znak {\tt \textasciitilde}) se píše mezi zkratkou jména a příjmením nebo mezi jednopísmennou předložkou/spojkou a~následujícím slovem. Mezi číslem a~jednotkou se píše {\bf úzká mezera}. Ukázka: {\tt 7.\textbackslash,11.\textbackslash,1811 se narodil K.\textasciitilde J.\textasciitilde Erben}.
\end{itemize}

\subsubsection*{Zvýrazňování textu}
Pro zvýraznění pojmů uvnitř textu se používá \textit{kurzíva} (příkaz \prikaz{textit} nebo lze využít makro \prikaz{ti} definované v~\uv{šabloně}).\par 
Silné zvýraznění (tj. tučné písmo) se v~závěrečných pracích uvnitř odstavců nepoužívá (nicméně existuje příkaz \prikaz{textbf} nebo makro \prikaz{tb} ze \uv{šablony}).


\subsubsection*{Psaní výčtů}
Pro psaní výčtů lze v~\LaTeX{u} využít prostředí {\tt enumerate} (číslovaný seznam) nebo {\tt itemize} (odrážky), položky píšeme do {\tt \textbackslash item}. Pravidla: \url{https://prirucka.ujc.cas.cz/?id=870}.


\subsubsection*{Psaní zkratek}
O~psaní zkratek pojednává heslo \uv{Zkratky, značky, čísla a~číslovky} na webovém rozcestníku \url{https://prirucka.ujc.cas.cz/}.

\subsubsection*{Složení čísel a slov}
Jak správně psát slova složená z~čísel a~slov? Vcelku podrobné vysvětlení naleznete na: \url{https://prirucka.ujc.cas.cz/?id=790}.\par 
Pozor na \textit{význam}: {\tt 5°} znamená pětistupňový, kdežto {\tt 5\,°} (s~úzkou mezerou) znamená pět stupňů.\par 
Ukázka správného zápisu (všimněte si: je to bez mezer a~bez spojovníku):\\
-- {\tt 20leté pozorování} nebo\\
-- {\tt dvacetileté pozorování} (jakékoli jiné tvary jsou chybné!).\par


\subsubsection*{Psaní značek}
Psaní značek, především matematických: \url{https://prirucka.ujc.cas.cz/?id=785}.


\section{Sazba matematického textu}

Mezi nejdůležitější typografické zásady psaní matematického textu patří:
\begin{itemize}\itemsep=2pt
\item známé konstanty a čísla určitá se píšou vzpřímeným řezem (např. $124$, $\mathrm e$, $\mathrm{\pi}$, $1-3\mathrm i$),
\item obecné konstanty a proměnné se píšou kurzívou (např. $a$, $x$, $x_1$),
\item identifikátory vektorů a matic se sázejí tučným řezem písma, např. $\mathbf{v}$, $\mathbf{M}$,
\item názvy funkcí (resp. známých polynomů) se píšou vzpřímeným řezem (např. $\sin(x)$, $\exp(y)$, $\mathrm{p}(x)$),
\item diferenciály se píšou vzpřímeným řezem, příslušná proměnná pak kurzívou: $\mathrm{d}x$.
\end{itemize}

Poznámka: příkazy pro matematickou sazbu jsou popsány též v~knize {\em \LaTeX pro začátečníky} od J.~Rybičky (ISBN 80-7302-049-1).


\subsection{Odkazy na rovnice nebo vztahy}
Rovnice, na které se budete v~textu odvolávat, opatřete pořadovými čísly při pravém okraji příslušného řádku (např. prostředí {\tt equation}). Příklad:
\begin{equation}\label{vzorec1}
p = \frac{1}{n}\cdot\sum_{i=1}^n c_i
\end{equation}

Číslování rovnic může být průběžné v~textu, nebo v~jednotlivých kapitolách, {\em odkaz na vzorec\/} vysází příkaz {\tt $\backslash$eqref\{odkaz\}} z~balíku {\tt amstex}, resp. napište: {\tt ($\backslash$ref\{odkaz\})}.
%Příklad: \hspace*{10mm} Průměrnou hodnotu z~čísel $c_1,c_2,\ldots,c_n$ vypočteme podle vztahu \eqref{vzorec1}.


\subsubsection{Desetinné číslo v~matematickém režimu} \label{des_cislo}
V českém textu používáme u~čísel desetinnou {\em čárku}, avšak čárka je v~mate\-ma\-tickém režimu
chápána jako oddělovač prvků seznamu ($\Rightarrow$ \LaTeX\ za ni automaticky přidává mezeru), proto je nutno desetinnou čárku uzavírat do složených závorek: {\tt \$a=21\{,\}7\$}.

\section{Ukázky zdrojových kódů}
Pokud chcete v~textu své práce upozornit na nějaký obzvláště zajímavý zdrojový kód\footnote{Veškeré zdrojové kódy svého programu odevzdáte jako samostatnou přílohu závěrečné práce v~systému KOS, typicky jako archiv (např. ZIP. Velikostní limit je asi 2~GB).}, můžete zařadit krátkou ukázku (obvyklejší však je, že ukázky zdrojového kódu uvedete až do textové přílohy práce s~patřičným komentářem. Nebo je neuvádíte do textu vůbec).

Prostředí pro zdrojové kódy: jednoduché {\tt verbatim} nebo pokročilejší {\tt lstlisting}, kde se zvýrazní syntaxe nastaveného programovacího jazyka (příkaz \prikaz{lstset}).


\section{Pravidla českého pravopisu}\label{sekcePravidla}
Text práce by se neměl prohřešit pravidlům českého pravopisu. Jako pomocník může posloužit např. \url{https://prirucka.ujc.cas.cz/}, kde nahoru napíšete slovo a necháte si jej vyhledat (zda vůbec takové slovo existuje a~jak se skloňuje/časuje, resp. jak se používá).
\par 
{\bf Tip:} nainstalujte si do editoru kontrolu {\em českého\/} pravopisu. Pokud si nejste čímkoli jisti, kupte/půjčte si aktuální \uv{Pravidla českého pravopisu} nebo požádejte o~kontrolu gramatických chyb osobu s~vytříbeným smyslem pro český jazyk (která se nebude zaměřovat na obsah práce, ale právě na stylistiku a pravopis).
\par
Pozor: není úkolem vedoucího práce, aby opravoval (a~četl) pravopisné chyby ve vašich \uv{betaverzích} textu! 


\section{Formátování seznamu použitých zdrojů}
% Stručně: https://prirucka.ujc.cas.cz/?id=883

Ve své závěrečné práci studenti pracují s~různými zdroji informací (ideálně s~knihami, nikoli nerecenzovanými weby). Text práce pak bude obsahovat údaje, které nejsou původní, a~proto je potřeba zdůraznit, co není studentův text:
\begin{itemize}
\item citují se použité {\em zdroje, které se týkají obsahu práce} (nikoli zdroje související s~úpravou/formou závěrečné práce, jako např. příručka k~\LaTeX u),
\item citován by měl být vždy {\em originální zdroj} (nikoli stránka z~Wikpedie). Pokud se jedná o~nějaký všeobecně známý fakt, normu, resp. standard de facto či de iure, tak není potřeba hledat zdroj;
\item seznam použitých zdrojů je umístěn za závěrem práce a~{\color{red} řadí se abecedně podle příjmení autorů} (pokud autor chybí, tak podle prvního slova).
\end{itemize}

Pravidla pro práci s~bibliografickými citacemi jsou definována českými technickými normami ČSN~ISO~690 a~690-2.
\par {\bf Tip:} použijte \url{https://www.citacepro.com/} a vyexportujte si uložené zdroje do TeXu (před exportem abecedně seřaďte dle příjmení autorů!)
\par 
Dbejte na {\em úplnost} použitých zdrojů $\Rightarrow$ uvádějte co nejvíce informací o~citovaném zdroji (např. název sborníku, číslo stránky ve sborníku, platný odkaz na článek). \par
Pozor: web CitacePro nemusí obsahovat relevantní informace o~použitém zdroji $\Rightarrow$ vždy si citaci zkontrolujte podle originální knihy!

%\cite{iso690,iso690-2}, z~nichž vycházejí také srozumitelnější formou psané dokumenty \cite{kra05,bold1,bold2,vym01}. 





\section{Křížové odkazy} \label{kap-odkazy}
Pro odkazy na stránky, na čísla kapitol a podkapitol, na čísla obrázků a tabulek atd. využíváme
speciálních prostředků DTP programu, které zajistí vygenerování správného čísla i~v případě, že se
text posune díky změnám samotného textu nebo díky úpravě parametrů sazby. V~systému \LaTeX\ jde o~odkaz na číslo, které odpovídá umístění značky v~textu (tzv. navěští).

\subsubsection{Definice návěští}

Návěští se definuje pomocí {\tt \textbackslash label\{navesti\}}\footnote{V seznamu použitých zdrojů se návěští definuje jinak: {\tt $\backslash$bibitem\{id\}}.\label{poznBibitem}}, uvedeného \underline{za} příkazem pro vytvoření objektu (např. u~obrázku za příkazem \prikaz{caption}, u~podkapitoly za příkazem \prikaz{section}).\par 
{\color{red} Návěští musí být v~rámci celého dokumentu {\em jedinečné}} -- na nejednoznačnosti upozorní {\LaTeX} při překladu dokumentu (+ zprávy se ukládají do LOG-souboru)!
\par 
Každé návěští se vztahuje k~nějakému čítači (coun\-ter), např. k~číslu kapitoly, podkapitoly, obrázku či tabulky. Můžete si také definovat vlastní prostředí
s~čítačem (příkaz \prikaz{newcounter}).

\subsubsection{Odkaz na návěští}
Použití odkazu závisí na tom, co potřebujete:
\begin{enumerate}
\item {\tt $\backslash$ref\{navesti\}} ... vysází odkaz na návěští, tj. vytiskne hodnotu příslušného čítače v~daném místě (např. {\em číslo} kapitoly nebo {\em číslo} obrázku).\par 
Pozor: příkaz netiskne název objektu(!) $\Rightarrow$ musíte napsat, od čeho je to číslo, např.: {\tt na {\color{red}obrázku}\textasciitilde \textbackslash ref\{obrXY\} je znázorněn průběh teplot};
%
\item {\tt$\backslash$pageref\{navesti\}} ... vysází \textit{číslo stránky}, kde se nachází odkazovaná položka;
\end{enumerate}
Speciality:
\begin{itemize}
\item {\tt $\backslash$eqref\{navesti\}} ... odkaz na číslo rovnice/vzorce (nutný balíček {\tt amsmath}); %
\item {\tt $\backslash$cite\{id\}} ... odkaz na položku ze seznamu použitých zdrojů$^{\ref{poznBibitem}}$.
\end{itemize}


\section{Finální úpravy před odevzdáním práce}

Závěrečná práce se odevzdává elektronicky v~systému KOS (termíny odevzdání jsou určeny harmonogramem akademického roku na FJFI).
\par
Před odevzdáním výsledného PDF si zkontrolujte (resp. nechte si od někoho zkontrolovat):
\begin{itemize}
\item zda je vloženo správné (podepsané) oficiální zadání práce + první strany práce mu odpovídají  ($\Rightarrow$ máte správně vyplněná makra z~tabulky~\ref{tabMakra});
\item jestli text práce neobsahuje gramatické (podkapitola~\ref{sekcePravidla}) nebo stylistické chyby,
\item zda v~práci nejsou typografické chyby (podkapitola~\ref{podkapTypo}). Pozor: často zůstávají na konci řádků jednopísmenné předložky/spojky,
\item jestli práce neobsahuje formální či věcné chyby (takovéto problémy pomůže odhalit vedoucí práce, kterému {\color{red} dostatečně včas\/} před termínem odevzdání práce {\color{red} několikrát} ukážete svůj text).
\end{itemize}




\begin{table}[h]
\centering
{\footnotesize
\caption{Makra definovaná v~\uv{šabloně} (v~abecedním pořadí).}\label{tabMakra}
\begin{tabular}{|l|l|l|}
\hline {\bfseries makro} & {\bfseries význam} & {\bfseries jak vyplnit} \\ 
\hline
\hline \prikaz{abstrCZ} & český popis práce (abstrakt) & vlastní text\\
\hline \prikaz{abstrEN} & anglický popis práce (abstract) & vlastní text\\ 
\hline \prikaz{autor} & \parbox{80mm}{\vspace{2px}\par jméno a příjmení autora závěrečné práce, včetně dosavadních titulů\par\vspace{3px}} & dle zadání! \\
\hline \prikaz{cvut} & oficiální název vysoké školy & dle zadání! \\
\hline \prikaz{druh} & typ závěrečné práce & dle zadání!\\
%\verb{\druh{Diplomová práce}} \\
\hline \prikaz{fjfi} & oficiální název fakulty & dle zadání!\\
\hline \prikaz{kde} & místo odevzdání (6. pád $\Rightarrow$ \uv{Praze} nebo \uv{Děčíně}) & místo studia\\
\hline \prikaz{keyword} & klíčová slova (anglicky) oddělená čárkou & vlastní text\\ 
\hline \prikaz{klicova} & klíčová slova (česky) oddělená čárkou & vlastní text\\
\hline \prikaz{konzultant} & jméno a příjmení konzultanta & dle zadání \\ % \footnotemark[\ref{pozn}]
\hline \prikaz{katedra} & oficiální název katedry & dle zadání!\\
\hline \prikaz{logoCVUT} & logo ČVUT (lev s~kružítkem) & neměnit!\\
\hline \prikaz{nazevcz} & český název práce & dle zadání!\\
\hline \prikaz{nazeven} & anglický název práce & dle zadání! \\
\hline \prikaz{podekovani} & text poděkování & vlastní text\\
\hline \prikaz{pracovisteKonz} & pracoviště konzultanta & dle zadání! \\
\hline \prikaz{pracovisteVed} & pracoviště vedoucího práce & dle zadání! \\
\hline \prikaz{program} & název studijního programu & dle zadání!\\
\hline \prikaz{prohlaseni} & text prohlášení & vlastní text\\
\hline \prikaz{rok} & konkrétní rok odevzdání práce (nikoli celý akad. rok) & dle skutečnosti\\
\hline \prikaz{spec} & název specializace (pokud ji váš studijní program má) & dle zadání! \\
\hline \prikaz{tb} & zvýraznění tučným písmem (zkratka příkazu \prikaz{textbf}) & neměnit\\
\hline \prikaz{ti} & zvýraznění kurzívou (zkratka příkazu \prikaz{textit}) & neměnit\\
\hline \prikaz{vedouci} & jméno a příjmení vedoucího práce, včetně titulů & dle zadání! \\
\hline \prikaz{woman} & koncovka minulého času u~sloves & \uv{a} pro ženy\\
%muž: \verb{\woman{}}, žena: \verb{\woman{a}} \\
\hline 
\end{tabular}
}
\end{table}

{\color{blue}Poznámka: případné chyby v~tomto textu pište na e-mail  {\tt dana.majerova(at)fjfi.cvut.cz}.}